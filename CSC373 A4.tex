\documentclass{article}
\usepackage[utf8]{inputenc}
\usepackage[a4paper, margin=0.9in, 10pt]{geometry}
\usepackage{amsmath, amssymb}
\usepackage{algorithm, algorithmic}
\usepackage{graphicx}
\usepackage{hyperref}
\usepackage{xcolor}
\usepackage{float}
\usepackage{amssymb}
\usepackage{xcolor}
\usepackage[makeroom]{cancel}

\title{CSC373 A4}
\author{Isaac Ng, Nathan Chao, Jean Li}
\date{December 7, 2020}

\newcommand{\bigO}{\mathcal{O}}

\begin{document}

\maketitle
\begin{enumerate}
    \item  
    \begin{enumerate}
        \item Note: In this problem, an edge from tower $n$ to tower $(n + 1)$ is an edge from tower $n$ to tower 1 and an edge from tower 1 to tower (1 - 1) is an edge from tower 1 to tower $n$.\\Binary Integer Linear Program solving this problem\\
        I want to define my two binary variables for this section.\\
        $forward_{k, x}$ is for request k and edge (x, x+1), 1 represents this cable being used while 0 shows its inactivity.\\
        Similarly, $backward_{k, x}$ is for request k and edge (x, x-1), 1 representing the cable being used and 0 being its inactivity.\\
        In a given request k, we are given (i, j, c) where i is the starting location, j is the final destination, and c is the byte size of the request.\\
        Here is the BILP:
        \begin{align*}
            min [ max_{x \in E} &( \sum_k c_k * forward_{k,x}, \sum_k c_k * backward_{k,x} ) ]\\
                forward_{k,i} &= forward_{k,i+1} = ... = forward_{k,j-1} \text{for each request k} \tag{1}\\
                backward_{k,i} &= backward_{k,i-1} = ... = backward_{k,j+1} \text{for each request k} \tag{2}\\
                forward_{k,j} &= forward_{k,j+1} = ... = forward_{k,i-1} = 0 \text{for each request k} \tag{3}\\
                backward_{k,j} &= backward_{k,j-1} = ... = backward_{k,i+1} = 0 \text{for each request k} \tag{4}\\
                forward_{k, i} &+ backward_{k,i} = 1 \tag{5} \text{for each request k} \\
                forward_{k,x}, &backward_{k,x} \in \{0, 1\} \tag{$\forall k, x$}
        \end{align*}
        
        \newpage
        \item 2-approximation relaxing our ILP\\
        Since there are k requests and n towers, we will be dealing with two n by k matrices of binary values when dealing with forward and backward.\\
        Consider the LP optimal solutions $forward^*$ and $backward^*$.\\
        Let $\hat{forward}_{k, x} = 1$ whenever $forward^* \geq 0.5$ and $\hat{forward}_{k, x} = 0$ otherwise.\\
        Let $\hat{backward}_{k, x} = 1$ whenever $backward^* > 0.5$ and $\hat{backward}_{k, x} = 0$ otherwise.\\
        \textbf{Claim 1: $\hat{forward}$ and $\hat{backward}$ is a feasible solution of the ILP.}\\
        Notice first, that statement (5) is satisfied. This is because in the LP, one value must be $\geq$ 0.5 while the other is $\leq$ 0.5. Therefore, one will round up to 1 while the other rounds down to 0. The case of both being 0.5 is taken care of by making one of the two variables $>$ instead of $\geq$.\\
        Also, from our rounding, the last statement holds true since all our variables round to 0 or 1.\\
        From our LP to the BILP, statements (3) and (4) stay the same since they had values set to 0 and so, after rounding, remain at 0.\\
        Statements (1) and (2) also stay the same since rounding the same number will give the same results.\\
        These two statements however, are used to determine the direction the request will travel.\\
        Notice that either $forward_{k, i}$ or $backward_{k,i}$ are set to 1 after our rounding. This means that for each request, only one of (1) and (2) will have a value of 0 and the other will have a value of 1.\\
        Therefore, these values provide a feasible solution since each request will travel in one direction from its starting location to its final destination.\\
        We have shown that $\hat{forward}$ and $\hat{backward}$ is a feasible solution of the ILP.\\
        \\
        \textbf{Claim 2: $max_{i \in E} &( \sum_k c_k * \hat{forward}_{k,i}, \sum_k c_k * \hat{backward}_{k,i} ) \leq 2 * max_{i \in E} &( \sum_k c_k * forward_{k,i}^*, \sum_k c_k * backward_{k,i}^* )$}\\
        The load on a cable only increases when some $forward_{k,x}^* \in [0.5, 1]$ is rounded up to 1 or when some $backward_{k,x}^* \in (0.5, 1]$ is rounded up to 1.\\
        This means that $forward_{k,x}^* \times 2 \geq \hat{forward}_{k,x}$ and $backward_{k,x}^* \times 2 \geq \hat{backward}_{k,x}$ for each request k and each tower x.\\
        This claim has been proven since each $forward_{k,x}^*$ and $backward_{k,x}^*$ are at most 2 times greater than $\hat{forward}_{k,x}$ and $\hat{backward}_{k,x}$.\\
        \\
         Therefore, since we know that relaxed LP's solution: $max[forward_{k,x}^*, backward_{k,x}^* ] \leq $ Optimal solution, we can say the following.\\
         $\hat{forward}, \hat{backward}$ is a solution to this problem with its maximum load being at most $2 \times$ LP optimal value $\leq 2 \times$ ILP optimal value.\\
         We have shown a 2-approximation by relaxing our ILP.
    \end{enumerate}
    
    % a) Note: In this problem, an edge from tower $n$ to tower $(n + 1)$ is an edge from tower $n$ to tower 1 and an edge from tower 1 to tower (1 - 1) is an edge from tower 1 to tower $n$.\\Binary Integer Linear Program solving this problem\\
    % I want to define my two binary variables for this section.\\
    % $forward_{k, x}$ is for request k and edge (x, x+1), 1 represents this cable being used while 0 shows its inactivity.\\
    % Similarly, $backward_{k, x}$ is for request k and edge (x, x-1), 1 representing the cable being used and 0 being its inactivity.\\
    % In a given request k, we are given (i, j, c) where i is the starting location, j is the final destination, and c is the byte size of the request.\\
    % Here is the BILP
    % \begin{align*}
    %     min [ max_{x \in E} &( \sum_k c_k * forward_{k,x}, \sum_k c_k * backward_{k,x} ) ]\\
    %         forward_{k,i} &= forward_{k,i+1} = ... = forward_{k,j-1} \text{for each request k} \tag{1}\\
    %         backward_{k,i} &= backward_{k,i-1} = ... = backward_{k,j+1} \text{for each request k} \tag{2}\\
    %         forward_{k,j} &= forward_{k,j+1} = ... = forward_{k,i-1} = 0 \text{for each request k} \tag{3}\\
    %         backward_{k,j} &= backward_{k,j-1} = ... = backward_{k,i+1} = 0 \text{for each request k} \tag{4}\\
    %         forward_{k, i} &+ backward_{k,i} = 1 \tag{5} \text{for each request k} \\
    %         forward_{k,x}, &backward_{k,x} \in \{0, 1\} \tag{$\forall k, x$}
    % \end{align*}
    
    % \\ b) 2-approximation relaxing our ILP\\
    % Since there are k requests and n towers, we will be dealing with two n by k matrices of binary values when dealing with forward and backward.\\
    % Consider the LP optimal solutions $forward^*$ and $backward^*$.\\
    % Let $\hat{forward}_{k, x} = 1$ whenever $forward^* \geq 0.5$ and $\hat{forward}_{k, x} = 0$ otherwise.\\
    % Let $\hat{backward}_{k, x} = 1$ whenever $backward^* > 0.5$ and $\hat{backward}_{k, x} = 0$ otherwise.\\
    % \textbf{Claim 1: $\hat{forward}$ and $\hat{backward}$ is a feasible solution of the ILP.}\\
    % Notice first, that statement (5) is satisfied. This is because in the LP, one value must be $\geq$ 0.5 while the other is $\leq$ 0.5. Therefore, one will round up to 1 while the other rounds down to 0. The case of both being 0.5 is taken care of by making one of the two variables $>$ instead of $\geq$.\\
    % Also, from our rounding, the last statement holds true since all our variables round to 0 or 1.\\
    % From our LP to the BILP, statements (3) and (4) stay the same since they had values set to 0 and so, after rounding, remain at 0.\\
    % Statements (1) and (2) also stay the same since rounding the same number will give the same results.\\
    % These two statements however, are used to determine the direction the request will travel.\\
    % Notice that either $forward_{k, i}$ or $backward_{k,i}$ are set to 1 after our rounding. This means that for each request, only one of (1) and (2) will have a value of 0 and the other will have a value of 1.\\
    % Therefore, these values provide a feasible solution since each request will travel in one direction from its starting location to its final destination.\\
    % We have shown that $\hat{forward}$ and $\hat{backward}$ is a feasible solution of the ILP.\\
    % \\
    % \textbf{Claim 2: $max_{i \in E} &( \sum_k c_k * \hat{forward}_{k,i}, \sum_k c_k * \hat{backward}_{k,i} ) \leq 2 * max_{i \in E} &( \sum_k c_k * forward_{k,i}^*, \sum_k c_k * backward_{k,i}^* )$}\\
    % The load on a cable only increases when some $forward_{k,x}^* \in [0.5, 1]$ is rounded up to 1 or when some $backward_{k,x}^* \in (0.5, 1]$ is rounded up to 1.\\
    % This means that $forward_{k,x}^* \times 2 \geq \hat{forward}_{k,x}$ and $backward_{k,x}^* \times 2 \geq \hat{backward}_{k,x}$ for each request k and each tower x.\\
    % This claim has been proven since each $forward_{k,x}^*$ and $backward_{k,x}^*$ are at most 2 times greater than $\hat{forward}_{k,x}$ and $\hat{backward}_{k,x}$.\\
    % \\
    %  Therefore, since we know that relaxed LP's solution: $max[forward_{k,x}^*, backward_{k,x}^* ] \leq $ Optimal solution, we can say the following.\\
    %  $\hat{forward}, \hat{backward}$ is a solution to this problem with its maximum load being at most $2 \times$ LP optimal value $\leq 2 \times$ ILP optimal value.\\
    %  We have shown a 2-approximation by relaxing our ILP.\\
    \newpage
    
    \item 
    \begin{enumerate}
        \item The algorithm schedules all tasks. A task $i$ is only scheduled at time $s_i$ which is the earliest time at which both worker $w_i$ and machine $m_i$ become available. This means each task is only scheduled when the required worker and machine are available. This satisfies all constraints.
        
        \item Assume for a contradiction that there is exists an idle worker that could be working on a yet-to-be-done task. This means that at time $t$, there exists an unscheduled task $i$ and that neither worker $w_i$ nor machine $m_i$ get used at time $t$. Since at time $t$, the algorithm uses neither worker $w_i$ nor $m_i$, at time $t$, $s_{i}= t$. Since the algorithm schedules neither worker $w_i$ nor  machine $m_i$ at time $t$, $s_i$ will never increase from $t$, so the algorithm will schedule $i$ to start at time $t$. This is a contradiction so there can never be an idle worker if she could be working on a yet-to-be-done task.
        
        \item Let $T$ be the overall finish time of the greedy schedule.\\
        Let $T^*$ be the minimum possible overall finish time of any schedule.\\
        Let $W_{\text{max}} = \max_jW_j$, where $W_j$ is the total processing time of all jobs $i$ with $w_i = j$.\\
        Let $M_{\text{max}} = \max_kM_k$, where $M_k$ is the total processing time of all jobs $i$ with $m_i = k$.\\
        Let $i$ be the task that finishes last in the schedule produced by the greedy algorithm. \\
        Let $W_{iwork}$ be the time spent working by worker $i$ up to $s_i$.  \\
        Let $W_{iidle}$ be the time spent idle by worker $i$ up to $s_i$. \\
        Let $M_{m_i}$ be the total processing time of all jobs $j$ with $m_j = {m_i}$.\\
        Fact (1): By part (b) and since $i$ is the last task, it follows that whenever worker $w_i$ is idle, $m_i$ is being used. This means $M_{m_i} \geq W_{iidle} + p_i$.
        \begin{align*}
        T &= s_i + p_i\\
        T &= W_{iwork} + W_{iidle}+ p_i \\
        T &\leq  W_\max + W_{iidle}+ p_i \\
        T &\leq  W_\max + M_{m_i} \tag{By fact (1)}\\
        T &\leq  W_\max + M_\max \\
        T &\leq  2\max(W_{\max}, M_{\max}) \\
        T &\leq  2T^* \\
        \end{align*}
        Therefore the algorithm achieves a 2-approximation.
    \end{enumerate}
    
%     q2\\a)\\
% The algorithm schedules all tasks. A task $i$ is only scheduled at time $s_i$ which is the earliest time at which both worker $w_i$ and machine $m_i$ become available. This means each task is only scheduled when the required worker and machine are available. This satisfies all constraints. \\
% \\
% b)\\
% Assume for a contradiction that there is exists an idle worker that could be working on a yet-to-be-done task. This means that at time $t$, there exists an unscheduled task $i$ and that neither worker $w_i$ nor machine $m_i$ get used at time $t$. Since at time $t$, the algorithm uses neither worker $w_i$ nor $m_i$, at time $t$, $s_{i}= t$. Since the algorithm schedules neither worker $w_i$ nor  machine $m_i$ at time $t$, $s_i$ will never increase from $t$, so the algorithm will schedule $i$ to start at time $t$. This is a contradiction so there can never be an idle worker if she could be working on a yet-to-be-done task.\\
% \\
% c)\\
% Let $T$ be the overall finish time of the greedy schedule.\\
% Let $T^*$ be the minimum possible overall finish time of any schedule.\\
% Let $W_{\text{max}} = \max_jW_j$, where $W_j$ is the total processing time of all jobs $i$ with $w_i = j$.\\
% Let $M_{\text{max}} = \max_kM_k$, where $M_k$ is the total processing time of all jobs $i$ with $m_i = k$.\\
% Let $i$ be the task that finishes last in the schedule produced by the greedy algorithm. \\
% Let $W_{iwork}$ be the time spent working by worker $i$ up to $s_i$.  \\
% Let $W_{iidle}$ be the time spent idle by worker $i$ up to $s_i$. \\
% Let $M_{m_i}$ be the total processing time of all jobs $j$ with $m_j = {m_i}$.\\
% Fact (1): By part b) and since $i$ is the last task, it follows that whenever worker $w_i$ is idle, $m_i$ is being used. This means $M_{m_i} \geq W_{iidle} + p_i$.
% \begin{align*}
% T &= s_i + p_i\\
% T &= W_{iwork} + W_{iidle}+ p_i \\
% T &\leq  W_\max + W_{iidle}+ p_i \\
% T &\leq  W_\max + M_{m_i} \tag{By fact (1)}\\
% T &\leq  W_\max + M_\max \\
% T &\leq  2\max(W_\max, M_\max) \\
% T &\leq  2T^* \\
% \end{align*}
% Therefore the algorithm achieves a 2-approximation.
    \newpage
    
    \item \begin{enumerate}
        \item We know the LP relaxation of the given ILP has an optimal solution $x^*$ in which $\forall v \in V, x^*_v \in \{0, \frac{1}{2}, 1\}$. We can convert $x^*$ to an integer solution $y$ by rounding $\frac{1}{2}$ up to 1. 
        
        Let $S'=\{v \in V: x^*_v = 1 \vee x^*_v = \frac{1}{2}$\}. Let $y$ be a solution to the given ILP where $y_v=1$ if $v \in S'$. Then $y$ meets both constraints of the given ILP. Therefore, $S'$ is a vertex cover of $G$ (but not necessarily a minimum vertex cover).
    
        From the constraints of the LP relaxation, we know that $x_u+x_v \geq 1, \forall (u, v) \in E$. \\Thus, $\forall (u, v) \in E, x^*_u=\frac{1}{2} \Rightarrow x^*_v \geq \frac{1}{2}$. This means that $\forall (u, v) \in E, (x^*_u=\frac{1}{2} \vee x^*_v=\frac{1}{2}) \Rightarrow u, v$ are both in $S'$.
        
        Note that $S$ is just $S'$, except with all vertices $v$ where $x^*_v=\frac{1}{2} \wedge c(v)=3$ removed. By the definition of 4-colouring, $\forall (u, v) \in E, c(u) \neq c(v)$, so at most one of $u$ or $v$ can belong to $V^{\frac{1}{2}}_3$. If one of $u$ or $v$ belongs to $V_3^{\frac{1}{2}}$, then the other must be in set $S$. 
        
        Thus, $S=\{v \in V: x^*_v=1 \vee (x^*_v=\frac{1}{2} \wedge v \not\in V^{\frac{1}{2}}_3)\}$ is still a vertex cover of $G$. \hfill $\blacksquare$
        
        \item Let $S^*$ be a minimum vertex cover of $G$. Let ${\color{red}S^k}= \{v \in V, x^*_v=k\}$ for $k \in \{0, \frac{1}{2}, 1\}$, where $x^*$ is an optimal solution to the LP relaxation of the given ILP.
        
        \begin{align*}
            |S^*| \geq \frac{1}{2}|{\color{red}S^\frac{1}{2}}| + |{\color{red}S^1}| \tag{claim (A), {\color{blue}proven at the bottom of the page}}
        \end{align*}
        
        Since $S^{\frac{1}{2}}$ is the union of $V_3^{\frac{1}{2}}, V_2^{\frac{1}{2}}, V_1^{\frac{1}{2}}, \text{ and } V_0^{\frac{1}{2}}$, and these sets do not have any overlap, and $|V_3^{\frac{1}{2}}| \geq |V_2^{\frac{1}{2}}| \geq |V_1^{\frac{1}{2}}| \geq |V_0^{\frac{1}{2}}|$, then 
        %
        \begin{align*} 
            |V_3^{\frac{1}{2}}| \geq \frac{1}{4}|S^{\frac{1}{2}}| \tag{B}
        \end{align*}
        %
        \begin{align*}
            |S|&=|S^1|+|S^{\frac{1}{2}}| - |V_3^{\frac{1}{2}}| \tag{by definition of $S$}\\
            & \leq |S^1| + \frac{3}{4}|S^{\frac{1}{2}}| \tag{by (B)}\\
            &= \frac{3}{2} (\frac{1}{2} |S^{\frac{1}{2}}| + |S^1|) - \frac{1}{2} |S^1|\\
            & \leq \frac{3}{2} |S^*| - \frac{1}{2} |S^1| \tag{by claim (A)}\\
            & \leq \frac{3}{2} |S^*|
        \end{align*}
        Since $|S|$ is at most $\frac{3}{2}$ the size of the smallest vertex cover, the algorithm achieves a \\ $\frac{3}{2}$-approximation. \hfill $\blacksquare$
        \\
        
        \emph{{\color{blue}Proof of claim}} {\color{blue}(A)}. 
        
        Let $S^*$ be a minimum vertex cover of $G$. Let ${\color{red}S^k}= \{v \in V, x^*_v=k\}$ for $k \in \{0, \frac{1}{2}, 1\}$, where $x^*$ is an optimal solution to the LP relaxation of the given ILP. We want to show $|S^*| \geq |{\color{red}S^1}| + \frac{1}{2}|{\color{red}S^{\frac{1}{2}}}|$. 
        
        Let $OPT(LP)$ and $OPT(ILP)$ be $\sum_{v \in V} x_v$ for an optimal solution to the linear program and integer linear program respectively. Since the linear program has a greater range of possible solutions,
        \begin{align*}
            OPT(LP) \leq OPT(ILP) = |S^*| \tag{C}
        \end{align*}
        
        Since each member of $S^1$ contributes a value of 1 to the sum and each member of $S^{\frac{1}{2}}$ contributes a value of $\frac{1}{2}$ to the sum of the LP's objective function, 
        \begin{align*}
            OPT(LP) = \sum_{v \in V}x^*_v = |S^1| + \frac{1}{2}|S^{\frac{1}{2}}| \tag{D}
        \end{align*}
         
        By (C) and (D), $|S^*| = OPT(ILP) \geq OPT(LP) = |S^1| + \frac{1}{2}|S^{\frac{1}{2}}|$. Therefore,
        \begin{align*}
            |S^*| \geq |S^1| + \frac{1}{2}|S^{\frac{1}{2}}|
        \end{align*}
        \hfill $\blacksquare$
    \end{enumerate} 
    \newpage
    
    \item 
    % maximize number of clauses j where sum of P_j > 0 or sum of N_j != |N_j|
    \begin{enumerate}
        \item \begin{align*}
            \text{maximize } & \sum_{j=1}^{m} z_j\\
            \text{subject to } & z_j \leq \sum_{i \in P_j} y_i + \sum_{i \in N_j} (1 - y_i) \text{ for all } j \in [m]\\
            & y_i, z_j \in \{0, 1\} \text{ for all } i \in \{1, ..., n\}, j \in \{1, ..., m\}
        \end{align*}
        
        \item For a clause $C_j$ to NOT be satisfied, all its literals must evaluate to FALSE. The probability that $C_j$ is not satisfied is $\Pi_{i \in P_j} (1 - f(y_i^*)) \times \Pi_{i \in N_j} f(y_i^*)$.
        %
        \begin{align*}
            f(y^*_i) & \geq 1 - 4^{(-y^*_i)} \tag{from definition of $f(y)$} \\
            1 - f(y^*_i) &\leq 1 - (1 - 4^{(-y^*_i)})\\
            &= 4^{(-y^*_i)}
        \end{align*}
        %
        \begin{align*}
            f(y^*_i) \leq 4^{y^*_i - 1} \tag{from definition of $f(y)$}\\
        \end{align*}
        From the above,
        \begin{align*}
            Probability(C_j \text{ is not satisfied}) &= \Pi_{i \in P_j} (1 - f(y_i^*)) \times \Pi_{i \in N_j} f(y_i^*) \\
            &\leq 4^{-\Sigma_{i \in P_j}(y^*_i)} \times 4^{-\Sigma_{i \in N_j}1 - y^*_i}\\
            &\leq 4^{-z_j^*} \tag{by constraint in LP in (a)}
        \end{align*}
        \hfill $\blacksquare$
        
        \item From the hint, we know $g(\lambda) \geq \lambda \times g(1)$ for all $\lambda \in [0, 1]$. 
        
        Note that $g(1)=1-\frac{1}{4}=\frac{3}{4}.$
        %
        \begin{align*}
            E[\text{number of clauses satisfied using }\mathcal{A}] &= \Sigma_{j=1}^{m} Probability(C_j \text{ is satisfied})\\
            &= \Sigma_{j=1}^{m} [1 - Probability(C_j \text{ is not satisfied})]\\
            & \geq \Sigma_{j = 1}^{m} 1 - 4^{-z^*_j} \tag{from part (b)}\\
            &\geq \Sigma_{j=1}^{m}g(1) \times z^*_j \tag{from the hint}\\
            &= \Sigma_{j=1}^{m}\frac{3}{4} \times z^*_j\\
            &= \frac{3}{4} \Sigma_{j=1}^{m}z^*_j\\
            &= \frac{3}{4} \times \text{optimal solution to LP relaxation}\\
            & \geq \frac{3}{4} \times \text{optimal solution to ILP} \tag{because ILP has a smaller range of possible solutions}\\
            &= 1 / \frac{4}{3} \times \text{optimal solution to ILP}\\
            & \
        \end{align*}
        Therefore, $\mathcal{A}$ is a 4/3-approximation algorithm. \hfill $\blacksquare$
    \end{enumerate}
\end{enumerate}

\end{document}
